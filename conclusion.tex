%!TEX root = ./main.tex

In this document we have looked at the properties of topological insulators, focussing on the example of the QWZ model -- a simple tight binding model with nearest neighbour interactions. We have introduced the definition of the Berry phase, Berry connection and Chern number, and seen how these can be used to classify the Hamiltonians of translationally symmetric quantum systems, as well understanding their role in the formation of gapless edge modes. Furthermore we have discussed the limitations of these concepts -- that they are only applicable to systems with periodic boundaries and translational symmetry. We introduce two efforts to extend the notion of Chern number to systems that do not fulfil these criteria, namely the Bott index and Chern number, and show how they are related. These quantities display some strange behaviour, such as being discontinuous where boundaries appear in the material, or showing some global rather than local properties. We have worked towards better understanding the causes of these discontinuities, as well as trying to understand whether they convey useful physical information. \par
Finally, we have introduced a set of tools that we hope will allow us to modify the definition of the Bott index to avoid the discontinuous behaviour at the edges. The modifications are based on the application of slowly varying magnetic flux to the material. This remains unfinished and it is not yet clear whether they will prove as useful as we hope. Many questions remain, firstly it is unclear whether adiabatic flux insertion is capable of providing us with stable operators that effect translation in $k$-space. Especially since any finite system will have some degree of coupling between opposing edge modes, allowing a left edge mode to transform into a right edge mode as $k$ is varied. It is not yet clear whether this will provide a significant obstacle to the creation of adiabatic operators that preserve left and right edge states. Furthermore the exact form of operators representing adiabatic flux insertion have not yet been calculated and tested to ensure that they have the right properties. In the short term the author intends to find the exact form of these operators and numerically test that they have the right properties and that an adiabatic Bott index using them is stable. This will be followed by efforts to write an analytic proof demonstrating the validity of this method. If this works, next steps will be to determine whether this stabilised Bott index has a physically significant interpretation. One candidate for this is to check wither it corresponds to a local definition of electric polarisation, since our current understanding of polarisation views it as a global quantity \cite{vanderbilt_berry_2018} -- in contrast to the classical picture. \par
In the longer term we have several further avenues to explore. One possibility is to examine whether these concepts can be used to determine the topological properties of interacting systems. It is not clear whether it is possible to create an equivalent Bott index that can act on a many-particle interacting ground state. Another possible avenue is to look at the behaviour of the Bott index in a system evolving dynamically. The author has found that the standard Bott index is unstable when applied to a quenched system similar to that used in \cite{caio_topological_2019}, so an interesting question could be to understand where this instability comes from and test whether it can be controlled by modifying the Bott index.
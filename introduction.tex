%!TEX root = ./main.tex

Following the discovery of the quantum Hall effect \cite{klitzing_new_1980} and its subsequent explanation \cite{laughlin_quantized_1981,thouless_quantized_1982} that introduced the notion of topological order, some of the most significant discoveries in condensed matter physics have come from the introduction of ideas from topology to the study of quantum systems. One such example are a class of materials known as topological insulators. These are insulating throughout the bulk, but are distinguished by the appearance of conducting edge states wherever a surface is present. Furthermore, the states often have unusual properties such as conducting unidirectionally or being unable to scatter off impurities in the material. Our understanding such materials has relied on using the geometric phase introduced by Berry \cite{berry_quantal_1984} to construct a set of topological invariants capable of classifying insulating Hamiltonians \cite{haldane_model_1988, bernevig_quantum_2006,kane_z_2005}. Throughout this document we focus on materials that can be classified according to the first Chern number. These classes are such that it is not possible to smoothly deform a Hamiltonian into a different class without crossing a point where the Hamiltonian becomes conductive, and a set of metallic states appear. A number of reviews on the subject can be found in \cite{hasan_topological_2010,qi_topological_2008,budich_adiabatic_2013}. \par
These methods have been effective at understanding a wide variety of materials \cite{chiu_classification_2016, cooper_topological_2019}, however calculating the Chern number of a material relies on the notion of momentum space, which is only well-defined when the material is translationally symmetric and lives in periodic boundary conditions. This presents a major obstacle to being able to understand the topological properties of materials with disorder, boundaries or other properties that disrupt the calculation such as interactions or dynamics. Recently, efforts have been made to find a quantity that is equivalent to Chern number, but only relies on local operators. Two such examples are the Bott index \cite{toniolo_equivalence_2017,loring_guide_2019} and Chern marker \cite{bianco_mapping_2011}. These are equivalent to the Chern number when the system satisfies the conditions for Chern number to be well-defined, however it is still possible to calculate them when Chern number cannot be calculated using the standard method. This allows us to comment on the topological properties of materials that were previously out of reach using the standard methods \cite{agarwala_topological_2017,huang_theory_2018,caio_topological_2019,toniolo_time-dependent_2018}. \par
In this review we will focus on understanding these topological markers. We start in \textsection\ref{sec:quantum_systems} by defining the types of quantum systems we will be looking at -- translationally symmetric tight-binding models -- and looking at the eigenstates of Hamiltonians in such systems. This is followed in \textsection\ref{sec:chern_number} by an explanation of the concept of Berry phase, Berry connection and Chern number. In \textsection\ref{sec:QWZ} we will look specifically at a single model, the Qi-Wu-Zhang model, and derive the form of its eigenstates in periodic and strip geometry. In \textsection\ref{sec:local_chern_markers} we introduce the Bott index and Chern marker, showing how they correspond to Chern number, as well as a number of example calculations allowing us to see how they behave in various systems. Finally in \textsection\ref{sec:strip_bott} we try to understand the reasons for some of the unexpected behaviour of the Bott index. We finish this by laying the foundations for a potentially improved definition of the Bott index. This definition is not yet complete, and we present only an intuitive justification for it as it is not yet clear whether the idea is sound.